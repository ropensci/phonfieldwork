\documentclass[13pt, t]{beamer}
% Presento style file
\usepackage{config/presento}

% custom command and packages
\input{config/custom-command}

\usepackage{color, colortbl}
\setlength{\columnseprule}{0.4pt} 

\title{\Large \hspace{-0.5cm} Phonetic fieldwork and experiments with \texttt{phonfieldwork} package for R}
\author[shortname]{George Moroz}
\institute[shortinst]{Linguistic Convergence Laboratory, NRU HSE, Moscow, Russia}
\date{\begin{center} {\large 15 November 2019} \bigskip \\ {Science seminar, \\ NRU HSE, School of linguistics, Moscow}\\ \vfill Presentation is available here: {\large \href{https://tinyurl.com/yzd2yjr5}{tinyurl.com/yzd2yjr5} \hfill \includegraphics[height = 2.5cm]{images/01_qrcode}} \end{center}}

\begin{document}

\begin{frame}[plain]
\maketitle
\end{frame}

\begin{frame}{Most phonetic research consists of the following steps:}
\begin{enumerate}
\item Formulate a research question. Think of what kind of data is necessary to answer this question, what is the appropriate amount of data, what kind of annotation you will do, what kind of statistical models and visualizations you will use, etc.
\item Create a list of stimuli.
\item Elicite list of stimuli from speakers who signed an Informed Consent statement, agreeing to participate in the experiment to be recorded on audio and/or video.
\item Annotate the collected data.
\item Extract the information from annotated data.
\item Create visualizations and evaluate your statistical models.
\item Report your results.
\item Publish your data. \pause
\end{enumerate}
The \texttt{phonfieldwork} package is created for helping with items 3, partially with 4, and 5 and 8.
\end{frame}

\begin{frame}{Why/when do you need the \texttt{phonfieldwork} package?}
These ideal plan hides some technical subtasks:
\begin{itemize}
\item creating a presentation for elicitation task
\item renaming and concatenating multiple sound files recorded during a session
\item automatic annotation in Praat TextGrids {\small \citep{boersma19}}
\item creating a searchable \texttt{.html} table with annotations, spectrograms and ability to hear sound
\item converting multiple formats (Praat, ELAN \citep{brugman04} and EXMARaLDA \citep{schmidt09}) \pause
\end{itemize}
\vfill
All of these tasks can be solved by a mixture of different tools:
\begin{itemize}
\item any programming language can handle automatic  file renaming
\item Praat contains scripts for concatenating and renaming files
\end{itemize}
\pause
\vfill
\texttt{phonfieldwork} provides a functionality that will make it easier to solve those tasks independently of any additional tools. You can also compare the functionality with other R packages:\\ \texttt{rPraat} \citep{borzil16} and \texttt{textgRid} \citep{reidy16}.
\end{frame}

\begin{frame}{Philosophy of the \texttt{phonfieldwork} package}
\begin{itemize}
\item each stimulus as a separate file
\item researcher carefully listens to consultants to make sure that they are producing the kind of speech they wanted
\item in case a speaker does not produce three clear repetitions, researcher ask them to repeat the task
\end{itemize}

There are some phoneticians who prefer to record everything, for language documentation purposes. I think that should be a separate task. If you insist on recording everything, it is possible to run two recorders at the same time: one could run during the whole session, while the other is used to produce small audio files. You can also use special software to record your stimuli automatically on a computer (e.g. \texttt{PsychoPy} \citep{peirce19}).
\end{frame}


\framecard[colorblue]{{\color{colorwhite} \Large Let's go through \textbf{phonfieldwork} functionality\pause :\\
\begin{itemize}
\item[\color{colorwhite} $\bullet$] \color{colorwhite} install the package
\item[\color{colorwhite} $\bullet$] \color{colorwhite} create a presentation based on a list of stimuli
\item[\color{colorwhite} $\bullet$] \color{colorwhite} rename collected data
\item[\color{colorwhite} $\bullet$] \color{colorwhite} merge all data together
\item[\color{colorwhite} $\bullet$] \color{colorwhite} automatically annotate your data
\item[\color{colorwhite} $\bullet$] \color{colorwhite} extract data from annotation
\item[\color{colorwhite} $\bullet$] \color{colorwhite} visualize your data
\item[\color{colorwhite} $\bullet$] \color{colorwhite} create an \textbf{.html} viewer
\item[\color{colorwhite} $\bullet$] \color{colorwhite} cite the package
\end{itemize}

}}

\begin{frame}[fragile]{Install the package}
\begin{itemize}
\item Install the package from CRAN:
\begin{verbatim}
install.packages("phonfieldwork")
\end{verbatim}
\vfill
\item \dots or install it from Github:
\begin{verbatim}
install.packages("devtools")
devtools::install_github("agricolamz/phonfieldwork") 
\end{verbatim} 
\pause \vfill
\item Load the package:
\begin{verbatim}
library("phonfieldwork")
packageVersion("phonfieldwork")

## [1] '0.0.3'
\end{verbatim}


\end{itemize}
\end{frame}

\begin{frame}[fragile]{Create a presentation based on a list of stimuli}
There are several ways to enter information about a list of stimuli into R:
\begin{itemize}
\item listing stimuli with the \texttt{c()}  function inside R
\begin{verbatim}
c("tip", "tap", "top")
[1] "tip" "tap" "top"
\end{verbatim}
\item importing \texttt{.csv} file inside R using the \texttt{read.csv()} function:
\begin{verbatim}
read.csv("my_stimuli_df.csv")
  stimuli vowel
1     tip     ı
2     tap     æ
3     top     ɒ
\end{verbatim}
\item importing data from \texttt{.xls} or \texttt{.xlsx} file inside R using the \texttt{read.csv()} function:
\begin{verbatim}
read.csv("my_stimuli_df.csv")
  stimuli vowel
1     tip     ı
2     tap     æ
3     top     ɒ
\end{verbatim}
\end{itemize}
\end{frame}

\begin{frame}[fragile]{Create a presentation based on a list of stimuli}
\begin{itemize}
\item Now we are ready for creating a presentation for elicitation:
\begin{verbatim}
create_presentation(stimuli = my_stimuli$stimuli,
                    output_file = "first_example",
                    output_dir = getwd()) 
\end{verbatim}
\href{https://agricolamz.github.io/phonfieldwork/first_example.html}{Here} is the result.
\item Here is another example with translations:
\begin{verbatim}
create_presentation(stimuli = my_stimuli$stimuli,
                    translation = my_stimuli$translation,
                    output_file = "first_example",
                    output_dir = getwd()) 
\end{verbatim}
\href{https://agricolamz.github.io/phonfieldwork/second_example.html}{Here} is the result.
\end{itemize}
\end{frame}

\begin{frame}{Rename collected data}

\end{frame}

\begin{frame}{Merge all data together}

\end{frame}

\begin{frame}{Automatically annotate your data}

\end{frame}

\begin{frame}{Extract data from annotation}

\end{frame}

\begin{frame}{Create an \texttt{.html} viewer}

\end{frame}

\begin{frame}{Cite the package}

\end{frame}

\framecard[colorblue]{{\color{colorwhite} \Large Send me a letter!\\
agricolamz@gmail.com\\ 
\vfill Presentation is available here: \\tinyurl.com/y3wtkcbq\\
\vfill \includegraphics[height = 4cm]{images/02_qrcode}}}

\begin{frame}{References}
\footnotesize
\bibliographystyle{config/chicago}
\bibliography{bibliography}
\end{frame}

\end{document}
